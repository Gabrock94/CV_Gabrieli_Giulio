\PassOptionsToPackage{dvipsnames}{xcolor}
\documentclass[10pt,a4paper]{altacv}


% Change the page layout if you need to
\geometry{left=1cm,right=9cm,marginparwidth=6.8cm,marginparsep=1.2cm,top=1.25cm,bottom=1.25cm,footskip=2\baselineskip}

% Change the font if you want to.

% If using pdflatex:
\usepackage[T1]{fontenc}
\usepackage[utf8]{inputenc}
\usepackage[default]{lato}

% If using xelatex or lualatex:
% \setmainfont{Lato}

% Change the colours if you want to
\definecolor{Navy}{HTML}{000080}
\definecolor{SlateGrey}{HTML}{2E2E2E}
\definecolor{LightGrey}{HTML}{666666}
\colorlet{heading}{Navy}
\colorlet{accent}{Navy}
\colorlet{emphasis}{SlateGrey}
\colorlet{body}{LightGrey}


% Change the bullets for itemize and rating marker
% for \cvskill if you want to
\renewcommand{\itemmarker}{{\small\textbullet}}
\renewcommand{\ratingmarker}{\faCircle}
%% sample.bib contains your publications
\addbibresource{biblio.bib}

\usepackage[colorlinks]{hyperref}

\usepackage{xstring}
\noexpandarg\exploregroups
\makeatletter
\let\orig@blx@bbl@entry\blx@bbl@entry
\def\blx@bbl@entry#1\endentry{%
	\StrSubstitute{#1}{Gabrieli}{\textbf{Gabrieli}}[\Result]%
	\expandafter\orig@blx@bbl@entry\Result\endentry}
\makeatother


\begin{document}
	
	\begin{fullwidth}
		\hspace*{\fill} {\footnotesize Last updated on \today}
	\end{fullwidth}
	
	\name{Giulio Gabrieli}
	\tagline{Postdoc}
	\personalinfo{%
		% Not all of these are required!
		% You can add your own with \printinfo{symbol}{detail}
		
		\email{\href{mailto:gack94@gmail.com}{gack94@gmail.com}}
		\email{giulio.gabrieli@iit.it}
		\email{giulio001@e.ntu.edu.sg}
	}
	
	%% Make the header extend all the way to the right, if you want. 
	\begin{fullwidth}
		\makecvheader
	\end{fullwidth}
	
	%% Depending on your tastes, you may want to make fonts of itemize environments slightly smaller
	\AtBeginEnvironment{itemize}{\small}
	
	
	%% Provide the file name containing the sidebar contents as an optional parameter to \cvsection.
	%% You can always just use \marginpar{...} if you do
	%% not need to align the top of the contents to any
	%% \cvsection title in the "main" bar.
	\cvsection[page1sidebar]{Education}
		\divider
		
	\cvevent{\textbf{PostDoc}}{Istituto Italiano di Tecnologia}{January 2022 -- Ongoing}{Rome, Italy}{}{}
	Center for Life Nano- \& Neuro-Science (CLNS) \\
	
	\divider
	\cvevent{\textbf{Ph.D. in Psychology}}{Nanyang Technological University}{January 2019 -- January 2022}{Singapore, Singapore}{}{}
	School of Social Sciences (SSS)\\
	\textbf{CGPA:} 4.83 / 5.0\\
	\textbf{GPA:} 3.90 / 4.0
	
	\divider
	
	\cvevent{MSc in Human-Computer Interaction}{University of Trento}{September 2016 - July 2018}{Trento, Italy}{}{}
	Department of Psychology and Cognitive Science (DIPSCO)\\
	Department of Information Engineering and Computer Science (DISI)\\
	\textbf{GPA:} 3.64 / 4.0\\
	\textbf{110/110 with honors}\\
	
	\divider
	
	\cvevent{BSc in Interfaces and Communication Technology}{University of Trento}{September 2013 - September 2016}{Trento, Italy}{}{}
	Department of Psychology and Cognitive Science (DIPSCO)\\
	\textbf{GPA:} 3.63 / 4.0\\
	\textbf{110/110 with honors}\\
	
	\divider
	
	\cvevent{Erasmus+}{Nagasaki University}{October 2016 - March 2017}{Nagasaki, Japan}{}{}
	Graduate School of Biomedical Sciences
	
	\divider
	
	\cvevent{Summer Research Internship}{Nanyang Technological University}{June 2016 - July 2016}{Singapore, Singapore}{}{}
	School of Humanities and Social Sciences
	
	
	\divider
	
	\cvevent{High School Diploma in Chemistry with specialization in Food Technology}{I.T.I.S. G. Ferraris}{September 2008 - June 2013}{Verona, Italy}{}{}
	
	\medskip
	
	\newpage
	\cvsection[page2sidebar]{Experience}
	
	\cvevent{Ph.D. Candidate}{Social and Affective Neuroscience Lab}{January 2019 - Ongoing}{Nanyang Technological University}{Signal processing (ECG, EDA, EMG, EEG, fNIRS, fMRI, infant cry), Experimental design \& Data collection, Big Data Analysis, Manuscript preparation}{}
	
	\divider
	
	
	\cvevent{Intern}{Affiliative Behaviour and Physiology Lab}{June 2015 - December 2018}{University of Trento}{Signal processing (ECG, EDA, EMG, EEG, infant cry), Experimental design \& Data collection, Manuscript preparation}{}
	
	\divider
	
	\cvevent{Data Entry}{Hellas Verona F.C. s.p.a}{June 2014 - December 2018}{Verona, Italy}{IT support and digital tickets management.}{}
	
	\divider
	
	\cvevent{Research assistant}{Affiliative Behaviour and Physiology Lab}{June 2018 - September 2018}{University of Trento}{Data analysis, manuscript preparation, speaker at scientific meetings.}{}
	
	\divider
	
	\cvevent{Scientific Meeting Organization}{University of Trento}{June 2017 - August 2017}{University of Trento}{Website design and IT support for 13th International Infant Cry Workshop.}{}
	

		
		\cvsection{Teaching Experience}
		\cvevent{Teaching Assistant}{HP2700: Abnormal Psychology}{2020-2021}{Nanyang Technological University, Singapore}
		\cvevent{Teaching Assistant}{HP4222-7237: The Neuroscience of Love
		}{2019-2020}{Nanyang Technological University, Singapore} 
		\cvevent{Teaching Assistant}{HP4021: Laboratory in Human and Animal Neuroscience}{2019-2020}{Nanyang Technological University, Singapore}
		\newpage
		\begin{fullwidth}
		\cvsection[]{Publications}
		
		\nocite{*}
		
		\def\yrlist{2022,...,2018}
		
		\defbibfilter{papers}{
			type=article or
			type=incollection or
			type=inproceedings or 
			type=inbook
		}
		
		\defbibfilter{talks}{
			type=presentation
		}
		
		\foreach \yr in \yrlist{
			\defbibcheck{publicationInthisYear}{
				\ifnumequal{\thefield{year}}{\yr}{}{\skipentry}
			}
			\printbibliography[check=publicationInthisYear, title=\yr,filter=papers]{}
		}
		
		\cvsection[]{Talks and Posters}
		
		\def\yrlist{2022,...,2016}
		\foreach \yr in \yrlist{
			\defbibcheck{publicationInthisYear}{
				\ifnumequal{\thefield{year}}{\yr}{}{\skipentry}
			}
			\printbibliography[check=publicationInthisYear, title=\yr, filter=talks]{}
		}
		
		\cvsection[]{Grants}
		
		\begin{itemize}
			\item Singapore's National Super Computing Centre January 2020 Cycle, Project: \textit{Computational study of Child Development in Low Resource Contexts} (Project ID: 12001609)
		\end{itemize}
		
		\cvsection[]{Thesis}
		\begin{itemize}
			\item \textbf{Gabrieli G.}, Esposito G. (2018), Using users' physiological response to predict aesthetic experience of websites, Master Degree in Human-Computer Interaction, University of Trento (Italy) [\href{http://www5.unitn.it/Biblioteca/it/Web/RichiestaConsultazioneTesi/364090}{PDF@UNITN}]
			
			\item \textbf{Gabrieli G.}, Esposito G. (2016), Development of a Web platform for graphical data collection and analysis, Bachelor Degree in Interfaces and Communication Technology, University of Trento (Italy) 
		\end{itemize}
		
		To get access to the above works, please \href{mailto:gack94@gmail.com}{contact me.}
		
		\cvsection[]{Software and Tools}
		\begin{itemize}
			\item \textbf{Pysiology} - A Python package for physyological's signals processing  [\href{https://github.com/Gabrock94/Pysiology}{GitHub}] [\href{https://pypi.org/project/pysiology/}{PyPI}] [\href{https://pysiology.rtfd.io}{Documentation}] 
			
			\item \textbf{pyaesthetics} -  Images' aesthetic analysis in Python (formerly known as PrettyWebsite)  [\href{https://github.com/Gabrock94/pyaesthetics}{GitHub}] [\href{https://pypi.org/project/pyaesthetics/}{PyPI}]] 
			
			\item \textbf{FNIRSQC} -  fNIRS Quality Control using Deep Learning Platform  [\href{https://socialaffectiveneuroscience.com/fnirsqc/}{Website}]
			
			\item \textbf{bfp}: a simple template to develop PWA with Firebase and Bootstrap.  [\href{https://github.com/Gabrock94/bfp}{GitHub}]
			
			\item \textbf{pyEasyTrend} - The simplest way to perform trend analysis in Python [\href{https://github.com/Gabrock94/pyEasyTrend}{GitHub}] [\href{https://pypi.org/project/pyEasyTrend/}{PyPI}] [\href{https://pyeasytrend.readthedocs.io}{Documentation}] 
		\end{itemize}
		
		\cvsection[]{Datasets}
		
		\def\yrlist{2021,...,2019}
		\foreach \yr in \yrlist{
			\defbibcheck{publicationInthisYear}{
				\ifnumequal{\thefield{year}}{\yr}{}{\skipentry}
			}
			\printbibliography[check=publicationInthisYear, title=\yr, type=dataset]
		}
		
		\cvsection[]{Preregistrations}
		\begin{itemize}
			
			\item \textbf{Gabrieli, G.}, Qi, S. N. S., Du Lim, N., \& Esposito, G.(2020, September 10). Halo Effect (Aesthetics x Trustworthiness) on objects on which Human Faces are displayed  (ATMs, short-term vs. long-term). \href{https://doi.org/10.17605/OSF.IO/BNPQZ}{https://doi.org/10.17605/OSF.IO/BNPQZ}
			
			\item \textbf{Gabrieli, G.}, Lim, Y. Y., Du Lim, N., \& Esposito, G. (2020, August 30). Halo Effect  (Aesthetics x Trustworthiness) of Human Faces under Priming conditions. \href{https://doi.org/10.17605/OSF.IO/RZDXH}{https://doi.org/10.17605/OSF.IO/RZDXH}
			
			\item \textbf{Gabrieli, G.}, \& Peipei, S. (2020, June 17). Automatic identification of toddlers’ pupillometry (fixation time) measures signal quality. Retrieved from \href{https://doi.org/10.17605/OSF.IO/ET95X}{https://doi.org/10.17605/OSF.IO/ET95X}
			
			\item Carollo A., Bizzego A., \textbf{Gabrieli G.}, Esposito G. (2020 May, 26) Artificial intelligence for the study of changes in human behavior during the COVID-19 outbreak. Retrieved from \href{https://osf.io/8xtbq/}{https://osf.io/8xtbq/}
			
			\item \textbf{Gabrieli, G.}, Esposito, G., Goh, R., \& Gualco, C. (2020, May 13). Effects of priming on faces' Aesthetic, Trustworthiness, Closeness, and Concern. Retrieved from \href{https://doi.org/10.17605/OSF.IO/YDS8B}{https://doi.org/10.17605/OSF.IO/YDS8B}
			
			\item \textbf{Gabrieli, G.}, Esposito, G., Goh, R., \& Gualco, C. (2020, May 13). Changes in individias' trusthworthiness judgments toward Asian faces following the 2019–20 coronavirus outbreak. Retrieved from \href{https://osf.io/hqj8n}{https://osf.io/hqj8n}
			
			\item Bonassi A., Carollo A., Cataldo I., \textbf{Gabrieli G.}, Lepri B., \& Esposito G. (2020, February 26). A potential interaction between genetic predispositions and adult attachment patterns on Facebook behaviour. Retrieved from \href{https://osf.io/egsc7}{https://osf.io/egsc7}
			
			\item Bonassi A., Cataldo I., \textbf{Gabrieli G.}, Lepri B., \& Esposito G. (2020, February 26). The interplay between Oxytocin Receptor Gene and adult close relationship on Instagram behaviour. Retrieved from \href{https://osf.io/t78fu}{https://osf.io/t78fu}
			
			\item Bonassi, A., Cataldo, I., \textbf{Gabrieli, G.}, Lepri, B., \& Esposito, G. (2019, November 20). Oxytocin Receptor Gene polymorphism and early parental bonding interact in modulating Facebook social behaviour. Retrieved from \href{https://osf.io/xudkn}{https://osf.io/xudkn}
			
			\item \textbf{Gabrieli, G.}, Esposito, G., \& Goh, R. (2019, October 30). Faces' Aesthetic and Trustworthiness. Retrieved from \href{https://osf.io/acpm7}{osf.io/acpm7}
			
			\item \textbf{Gabrieli G.}, Goh R., Esposito G., \& Carollo A. (2019, October 13). Baby faces and Media. Retrieved from \href{https://osf.io/4n6mh}{osf.io/4n6mh} 
			
			\item Bonassi, A., Cataldo, I., \textbf{Gabrieli, G}., Lepri, B., \& Esposito, G. (2019, October 4). Oxytocin Receptor Gene and early parental bonding interact in shaping Instagram social behaviour. Retrieved from \href{https://osf.io/j9nqc}{https://osf.io/j9nqc} 
			
		\end{itemize}
		
		\cvsection[]{Peer Review Summary}
		\begin{itemize}
			\item Research in Developmental Disabilities (11)
			\item Frontiers in Psychology (10)
			\item International Journal of Environmental and Public Health (10)
			\item Acta Psychologica (2)
			\item Behavior Research Methods (2)
			\item Behavioral Sciences (2)
			\item Infant Behavior and Development (2)
			\item Concurrency and Computation: Practice and Experience (1)
			\item Frontiers in Public Health (1)
			\item Inquiry (1)
			\item Scientific Data (1)
			\item Smart Innovation, Systems and Technologies (1)
			\item Swiss National Science Foundation (1)
		\end{itemize}
		
		
		\cvsection[]{Press Coverage}
		
		\begin{itemize}
			\item 'Beauty', Drive Time. Voice of Islam Radio (2022, July 28). \url{https://soundcloud.com/voislam/drive-time-show-podcast-28-07-2022-beauty-and-poverty}
			\item Scientificast, Episode 384 (2021, November 29). \url{https://www.scientificast.it/psicofisiologia-dei-coralli-del-tempo/}
			\item Infancy Research Around the World: Singapore. International Congress of Infant Studies (2021, September 30). \url{https://infantstudies.org/infancy-research-around-the-world-singapore/}.
			\item Could oxytocin receptors influence Instagram activity?. Medical News Today (2021, September 23). \url{https://www.medicalnewstoday.com/articles/could-oxytocin-receptors-influence-instagram-activity}
			\item Variations in ‘love hormone’ gene might affect how much people post on Instagram, researchers suggest. The independent (2021, September 22). \url{https://www.independent.co.uk/life-style/gadgets-and-tech/instagram-followers-gene-love-hormone-b1924800.html}
			\item A Possible Connections Between Oxytocin and Instagram. Neuroscience.com (2021, September 22). \url{https://neurosciencenews.com/oxytocin-genetics-instagram-19340/}
			\item Researchers look for possible connections between oxytocin and Instagram. Eurekalert.org (2021, September 22). \url{https://www.eurekalert.org/news-releases/928645}
			\item Why Genuine, Lasting Connections Feel So Elusive. Psychology Today (2021, May 4). \url{https://www.psychologytoday.com/za/blog/missing-each-other/202105/why-genuine-lasting-connections-feel-so-elusive}
			\item Parents’ brain activity synchronizes in each other’s presence. Medical News Today (202, May 23). \url{https://www.medicalnewstoday.com/articles/parents-brain-activity-synchronizes-in-each-others-presence}
			\item Parents’ brains sync when taking care of children together. The Star (2020, May 22). \url{https://www.thestar.com.my/lifestyle/health/2020/05/22/parents-brains-sync-when-taking-care-of-children-together}
			\item Here’s how co-parenting with one’s spouse can alter each other’s brain activity, improve relations. Hindustan Times (2020, May 18). \url{https://www.hindustantimes.com/sex-and-relationships/here-s-how-co-parenting-with-one-s-spouse-can-alter-each-other-s-brain-activity-improve-relations/story-CJqlk1U7pXlncWaqL1TZdO.html}
			\item Parents’ brains sync up when caring for children together. Big Think (2020, May 15). \url{https://bigthink.com/neuropsych/parents-brain-changes/}
			\item Spouses Can Alter Each Other’s Brain Activity When Co-Parenting. SciTechDaily (2020, May 13). \url{https://scitechdaily.com/spouses-can-alter-each-others-brain-activity-when-co-parenting/}
			\item Physical presence of spouses can alter each other's brain activity. News-Medical (2020, May 12). \url{https://www.news-medical.net/news/20200512/Physical-presence-of-spouses-can-alter-each-others-brain-activity.aspx}
			\item Presence of spouse alters how parents' brains react to children stimuli. Science Daily (2020, May 12). \url{https://www.sciencedaily.com/releases/2020/05/200512100236.htm}
			\item Physical presence of spouse alters how parents' brains respond to stimuli from children. Medical Xpress (2020, May 12). \url{https://medicalxpress.com/news/2020-05-physical-presence-spouse-parents-brains.html}
			\item Presence of spouse alters how parents' brains react to children stimuli, finds NTU Singapore study. Eurekalert.org (2020, May 12). \url{https://www.eurekalert.org/news-releases/868472}
			\item In Times of Joy and in Times of Sorrow: The Complex Intricacies Behind Co-Parenting and Brain-to-Brain Synchrony. Thrive Global (2020, May 11). \url{https://thriveglobal.com/stories/in-times-of-joy-and-in-times-of-sorrow-the-complex-intricacies-behind-co-parenting-and-brain-to-brain-synchrony/}
			\item Stressing About Your Parenting Ability Might Be Counterproductive. Thrive Global (2019, September 12. \url{https://thriveglobal.com/stories/stressing-about-your-parenting-ability-may-be-counterproductive/}
			\item Parenting stress weakens mother-child communication. The statesman (2019, August 31). \url{https://www.thestatesman.com/lifestyle/parenting-stress-weakens-mother-child-communication-1502794316.html}
			\item Parental stress can affect communication with children. Deccan Chronicle (2019, August 31). \url{https://www.deccanchronicle.com/lifestyle/sex-and-relationship/310819/parental-stress-can-affect-communication-with-children.html}
			\item Parental stress can affect communication with children. The Asian Age (2019, August 31). \url{https://www.asianage.com/life/relationship/310819/parental-stress-can-affect-communication-with-children.html}
			\item Parenting stress weakens mother-child communication. The Siasat Daily (2019, August 31). \url{https://www.siasat.com/parenting-stress-weakens-mother-child-communication-1601460/}
			\item Parenting stress may weaken mother-child communication, study shows. News Medical (2019, August 29). \url{https://www.news-medical.net/news/20190829/Parenting-stress-may-weaken-mother-child-communication-study-shows.aspx}
			\item Parenting stress may affect mother and child ability to tune in to each other. Medical Xpress (2019, August 29). \url{https://medicalxpress.com/news/2019-08-parenting-stress-affect-mother-child.html?src_id=alt}
			\item Parenting stress may affect mother's and child's ability to tune in to each other. Science Daily (2019, August 29). \url{https://www.sciencedaily.com/releases/2019/08/190829101059.htm}	
			
		\end{itemize}
		
		\cvsection{Non-Formal Education}
		\cvevent{NIRx Virtual fNIRs Training Course}{NIRx}{January 25 - February 5 2021}{Online}
		\cvevent{HTML5 Game Development with Phaser}{Linkedin Learn}{November 2020}{Online}{}{}
		\cvevent{Cloud OnBoard: GCP Fundamentals Series}{Google Cloud}{24–26 March 2020}{Online}{}{}
		\cvevent{MATLAB Onramp}{MathWorks}{January 2019}{Online}{}{}

		
		\newpage
		
		\cvsection{References}
		
		\begin{itemize}
			\item \textbf{Assoc. Prof. Gianluca Esposito}\\
			Department of Pscyhology and Cognitive Science, University of Trento, Italy\\
			\href{mailto:gianluca.esposito@unitn.it}{gianluca.esposito@unitn.it}

			\item \textbf{Dr. Andrea Bizzego}\\
			Department of Pscyhology and Cognitive Science, University of Trento, Italy\\
			\href{mailto:andrea.bizzego@unitn.it}{andrea.bizzego@unitn.it}
			
			\item \textbf{Dr. Anna Truzzi}\\
			Trinity College Institute of Neuroscience, Trinity College, Ireland\\
			\href{mailto:truzzia@tcd.ie}{truzzia@tcd.ie}

		\end{itemize}
		
	\end{fullwidth}
	
\end{document}
